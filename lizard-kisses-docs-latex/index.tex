\documentclass[a4paper,12pt]{article}
\usepackage[T1]{fontenc}
\usepackage{ninecolors}
\usepackage{booktabs}
\usepackage{caption}
\usepackage{tabularray}
\usepackage{geometry}
% \usepackage[margin=2cm]{geometry}
\captionsetup[table]{position=bottom}

\begin{document}
\begin{titlepage}
  \vspace*{\stretch{1.0}}
  \begin{center}
    \Large\textbf{Build Lizard Kisses}\\
    \large{Soldering Workshop by Pedal Markt}
  \end{center}
  \vspace*{\fill}
  \begin{center}
    \today
  \end{center}
\end{titlepage}
\section{BOM – Bill of Materials}

Here is a list of parts you'd need to build the pedal.
Each row corresponds to a component with a certain value,
for example a `ceramic capacitor with value 1nF.` There could
be one or more actual physical parts per each row,
their designators are listed in the \textit{Reference}
column.

\newgeometry{top=1cm}

\begin{table}[h!]
  \centerline{
    \begin{tblr}{
      hlines,
      vlines,
      rows={ht=1.2em},
      row{odd}={bg=gray9},
      width=1.3\linewidth,
      colspec={lX[1]llX[2]}
    }
      \SetCell[c=5]{c}\textbf{Main board, floor side}
      \\
      \hspace{1em}
      & \textbf{Ref}
      & \textbf{Value}
      & \textbf{Qnty}
      & \textbf{Description}
      \\
      \hspace{1em}
      & GND & Wire & 2 & $\approx10cm$, strip and pre-tin both ends
      \\
      \hspace{1em}
      & IN & Wire & 1 & $\approx10cm$, strip and pre-tin both ends
      \\
      \hspace{1em}
      & OUT & Wire & 1 & $\approx10cm$, strip and pre-tin both ends
      \\
      \hspace{1em}
      & D2 & 1N4148 & 1
      & Diode
      \\
      \hspace{1em}
      & R7 & 4.7k & 1 & Resistor
      \\
      \hspace{1em}
      & R1 & 1k & 1
      & Resistor for the LED
      \\
      \hspace{1em}
      & R12, R13 & 1k & 2
      & Resistor
      \\
      \hspace{1em}
      & R6, R10 & 2.2k & 2
      & Resistor
      \\
      \hspace{1em}
      & R2, R5, R8 & 1M & 3
      & Resistor
      \\
      \hspace{1em}
      & R3, R4, R9, R11, R14, R15, R16 & 10k & 7
      & Resistor
      \\
      \hspace{1em}
      & Q2, Q3 & 3-pin socket  & 2
      & For transistor
      \\
      \hspace{1em}
      & C6, C7 & 2-pin socket & 2
      & For capacitor
      \\
      \hspace{1em}
      & Diode Pairs & 4-pin socket & 4
      & For clipping diodes
      \\
      \hspace{1em}
      & Q1 & TP0606 & 1
      & P-channel MOSFET
      \\
      \hspace{1em}
      & Q5 & 2N3906 & 1
      & PNP transistor
      \\
      \hspace{1em}
      & Q4, Q6 & 2N3904 & 2
      & NPN transistor
      \\
      \hspace{1em}
      & C5, C8 & 47p & 2
      & Ceramic capacitor
      \\
      \hspace{1em}
      & C3 & 47n & 1
      & Film capacitor
      \\
      \hspace{1em}
      & C9 & 100n & 1
      & Film capacitor
      \\
      \hspace{1em}
      &  & Power Socket & 1
      & 2-pin JST on the bottom-left of the board
      \\
      \hspace{1em}
      & C4, C10, C11 & 1u & 3
      & Film capacitor
      \\
      \hspace{1em}
      & C1 & 100u & 1
      & Electrolytic capacitor
      \\
      \hspace{1em}
      & C2 & 47u & 1
      & Electrolytic capacitor
      \\
      \SetCell[c=5]{c}\textbf{Outboard}
      \\
      \hspace{1em}
      & & DC Jack & 1
      & Mount and wire up the DC jack
      \\
      \hspace{1em}
      & & Audio Jack & 2
      & Wire up audio jacks
      \\
      \SetCell[c=5]{c}\textbf{Main board, player side}
      \\
      \hspace{1em}
      &  & Ribbon cable & 1
      & On the bottom-center of the board
      \\
      \hspace{1em}
      & VOL, GAIN  & A100k & 2
      & Potentiometers
      \\
      \hspace{1em}
      & BRIGHT & On-On & 1
      & Switch
      \\
      \hspace{1em}
      & CLIP & On-Off-On & 1
      & Switch
      \\
      \hspace{1em}
      & & LED & 1
      &
      \\
      \SetCell[c=5]{c}\textbf{Switch board, player side}
      \\
      \hspace{1em}
      &  & Footswitch & 1
      & Switch
      \\
    \end{tblr}
  }
  \caption{BOM}
\end{table}

\restoregeometry

\end{document}
